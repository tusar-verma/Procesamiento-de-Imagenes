\documentclass{article}

\usepackage{graphicx} % Required for inserting images
\usepackage{caratula} % Version modificada para usar las macros de algo1 de ~> https://github.com/bcardiff/dc-tex

\graphicspath{ {./} }
\usepackage{amsmath}
\usepackage[left=1.5cm, right=2cm]{geometry}
\geometry{
 a4paper,
 total={170mm,257mm},
 left=20mm,
 top=20mm,
 right=20mm
 }


\begin{document}
\titulo{Trabajo Final}
\subtitulo{Image mosaicing using Harris corner detection}
\fecha{2do cuatrimestre 2024}
\materia{Procesamiento de Imágenes}
\grupo{Grupo 3}

\newcommand{\dato}{\textit{Dato}}
\newcommand{\individuo}{\textit{Individuo}}

% Pongan cuantos integrantes quieran
\integrante{Olivarez Zigarán, Víctor Vicente}{443/22}{victorolivarez2001@gmail.com}
\integrante{Torrez, Matías Nicolás}{625/22}{matiastorrez157@gmail.com}
\integrante{Gonzalez, Maximiliano Javier}{659/22}{gonzalezmaxijavier@gmail.com}

\maketitle

\section{Introducción}
\par Hola
% p
% \
\section{Deteccion de esquinas con Harris}
Utilizaremos el algoritmo de Harris para detectar las esquinas en las dos imágenes que
queremos unir. 
\section{Homografia}
\par En el contexto de la unión de imágenes medicas, las dos imágenes a unir se pueden asumir fueron tomadas desde ángulos
similares sobre un mismo objeto, que lo pensaremos como un plano $\pi$ . Llamaremos a las dos cámaras C1 y C2 que apuntan
a nuestro plano original $\pi$. Sea un punto $P$ perteneciente a nuestro plano original, tomaremos las proyecciones de 
este punto sobre los planos captados por C1 y C2.

\par En particular, a cada uno de estos puntos de nuestro plano $\pi$ los pensaremos como puntos en tres coordenadas 
con valores $P_1$ (u1,v1,w1) y $P_2$ (u2,v2,w2). Sin embargo, para simplificar cálculos, pensaremos que el plano de nuestras 
cámaras C1 y C2 están a una distancia 1 del plano xy, sobre el cual estara $\pi$. Entonces, usando coordenadas
homogéneas $P_1$ = (u1', v1', 1) y $P_2$ = (u2', v2', 1), con u1' = u1/w1, 
u2'= u2/w2, v1'= v1/w1 y v2'= v2/w2.

\begin{figure}[h]
    \centering
    \includegraphics[width=0.5\linewidth]{HomogeneusCoords.png}
    \caption{Proyecciones de los planos de las cámaras sobre el plano $\pi$}
    \label{fig:homogeneus-coords}
\end{figure}

Entonces, sean $P_1$ = $(u1',v1',1)^t$ y $P_2$ = $(u2',v2',1)^t$, podemos pensar que existe una matriz $H$ tal que
para todo punto $P\in \pi$ cuyas proyecciones sean $P_1,P_2$ se cumple lo siguiente
\begin{figure}[h]
    \centering
    \begin{equation}
        P_1 = HP_2
    \end{equation}
    %\caption{Proyecciones de los planos de las cámaras sobre el plano $\pi$}
    \label{fig:Hmatrix}
\end{figure}

Esto es debido a que los estamos pensando en coordenadas homogéneas, donde un vector o matriz $v$ es equivalente $v k$ con $k \neq 0$ 
\begin{figure}[h]
    \centering
    \begin{equation}
        \begin{bmatrix}
            u1' \\
            v1' \\
            1   \\
        \end{bmatrix}
        =
        \begin{bmatrix}
            h_{11} & h_{12} & h_{13} \\
            h_{21} & h_{22} & h_{23} \\
            h_{31} & h_{32} & h_{33} 
        \end{bmatrix}
        \begin{bmatrix}
            u2' \\
            v2' \\
            1   \\
        \end{bmatrix}
    \end{equation}
    \label{fig:H-bmatrix}
\end{figure}

Luego, nos podemos aprovechar de que multiplicar a $H$ por $1/h_{33}$ no modifica la ecuación a fines de definir nuestro $h_{33}' = 1$ y así obtener que nuestra matriz tenga 8 grados de libertad.
Sin embargo esto nos trae un problema, y es que si $h_{33} = 0$, este método no nos permitirá hallar $H$, que es el objetivo de nuestro trabajo. Otra forma de forzar los 8 grados de libertad es normalizar la matriz tal que $|| (h_{11},h_{12},h_{13},h_{21},h_{22},h_{23},h_{31},h_{32},h_{33} )|| = 1$.
% Aca falta explicar un poco mas xq eso vale
El paso siguiente es hallar los valores de H. Para esto planteareamos las siguientes ecuaciones
\begin{figure}[]
    \centering
    \begin{equation}
        u_1' = (h_{11} u_2' + h_{12} v_2' + h_{13})/(h_{31} u_2' + h_{32} v_2' + h_{33})
    \end{equation}
    %\caption{Proyecciones de los planos de las cámaras sobre el plano $\pi$}
    \label{fig:H-eq-1}
\end{figure}
\begin{figure}[]
    \centering
    \begin{equation}
        u_2' = (h_{21} u_2' + h_{22} v_2' + h_{23})/(h_{31} u_2' + h_{32} v_2' + h_{33})
    \end{equation}
    %\caption{Proyecciones de los planos de las cámaras sobre el plano $\pi$}
    \label{fig:H-eq-1}
\end{figure}

Luego despejando 
\begin{figure}[h]
    \centering
    %\begin{equation}
    \begin{align}
        (h_{31} u_2' + h_{32} v_2' + h_{33})u_1' = (h_{11} u_2' + h_{12} v_2' + h_{13})\\
        (h_{31} u_2' + h_{32} v_2' + h_{33})u_2' = (h_{21} u_2' + h_{22} v_2' + h_{23})\\
    \end{align}
    %\end{equation}
    %\caption{Proyecciones de los planos de las cámaras sobre el plano $\pi$}
    \label{fig:H-eq-2}
\end{figure}
 y finalmente
\begin{figure}[h]
    \centering
    %\begin{equation}
    \begin{align}
       h_{11} u_2' + h_{12} v_2' + h_{13} - h_{31} u_2' u_1' - h_{32} v_1' u_2' - h_{33} u_2'\\
        (h_{31} u_2' + h_{32} v_2' + h_{33})u_2' = (h_{21} u_2' + h_{22} v_2' + h_{23})\\
    \end{align}
    %\end{equation}
    %\caption{Proyecciones de los planos de las cámaras sobre el plano $\pi$}
    \label{fig:H-eq-2}
\end{figure}
\end{document}
